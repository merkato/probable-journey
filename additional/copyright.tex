%!TEX root = ../skrypt-kierownik.tex

\clearpage

\begin{adjustwidth}{0.23\textwidth}{0.23\textwidth}
\begingroup
  \null\vfill
  \begin{center}
  \utitle\par
  Wydanie \uedition, wersja \urevision\par
  Copyright \copyright{} \udate\ by \uauthor\\
  Utwór nie może być powielany i rozpowszechniany, w jakiejkolwiek formie
  i w jakikolwiek sposób, bez pisemnej zgody autora.\par  
  

Konsultacja merytoryczna: Ryszard Grzecznik, Andrzej Pietras\\
%Korekta: \\
Strona tytułowa: Michał Siewior\\
Ilustracje: wykorzystano grafiki z serwisu pl.Wikipedia - \ref{fig:rozjazd}, \ref{fig:adr}, \ref{fig:wozek}, Beskidzka Strona Kolejowa  - \ref{fig:siec}, \ref{fig:numeracja-torow}, PKP.REPO - \ref{fig:wskazniki}, Tomasz Herud (Therud) (Wikipedia) - \ref{fig:wskaznikw11a}, Swisstack (Wikipedia) - \ref{fig:sygnalizatory}, Chemet - \ref{fig:cysterna}, TransportSzynowy.pl - \ref{fig:pantograf1}, \ref{fig:pantograf2}, ''Drogi Szynowe'' - \ref{fig:tory}, http://wolf.ict.pwr.wroc.pl/covalus - \ref{fig:strefa}, \ref{fig:przewod}, 
\\pozostałe ilustracje Tomasz Nycz
\par
  
  Wydanie pierwsze 2019\par
  Wydawca \upublisher.\par
  Podręcznik ten nie jest oficjalnym dokumentem spółki Koleje Śląskie Sp. z o.o., wykorzystanie danych zawartych w opracowaniu wyłącznie na własną odpowiedzialność.\par

  \uwebsite
  \end{center}
  \vspace*{10mm}
\endgroup
\end{adjustwidth}

\clearpage